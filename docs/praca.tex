\documentclass[a4paper]{article}
\usepackage[pdftex]{graphicx}
% oprócz formatu papieru zdefiniowanego na początku (A4) możemy
% jeszcze określić wielkość „obszaru zadrukowanego”:
\usepackage[body={17cm,24cm}]{geometry}
%%%%%%%%%%%%%%%%%%%%%%%%%%%%%%%%%%%%%%%%%%%%%%%%%%%%%%%%%%%%
% aby ustalić, że plik źródłowy używa kodowania UTF-8:
%-----------------------------------------------------------
\usepackage[utf8]{inputenc}
% jeśli jedynie chcemy wykorzystywać polskie (lub inne) 
% znaki diakrytyczne wystarczy użyć pakietu "fontenc"
% z parametrem "OT4":
%\usepackage[OT4]{fontenc}
% jeśli piszemy po polsku, należy zamiast powyższego użyć:
\usepackage{polski}
%%%%%%%%%%%%%%%%%%%%%%%%%%%%%%%%%%%%%%%%%%%%%%%%%%%%%%%%%%%%
\begin{document}
\section{Streszczenie}
% Poniższy akapit zawiera przykłady użycia „niełamliwej spacji” w celu uniknięcia
% „wiszących liter” na końcach wiersza – zjawisko uważne za błąd w polskiej typografii.
Poniższa praca zawiera opis biblioteki „Ash” służącej do funkcjonalnego testowania
hybrydowych aplikacji mobilnych stworzonych przy użyciu Adobe PhoneGap lub
Apache Cordova. Ash pozwala na testowanie zachowania się aplikacji w~różnorodnych
realistycznych scenariuszach, które bywają trudne do symulowania przez inne narzędzie, takich jak poruszanie się użytkownika,
obrót ekranu, utrata dostępu do sieci oraz~inne. Dzięki wykorzystaniu hybrydowego charakteru aplikacji
możliwa jest emulacja zachowania, co wpływa na większy realizm testów, a tym samym na wiarygodność ich wyników. Innymi
zaletami bliblioteki są elastyczna struktura testów, która ułatwia utrzymywanie
testów i budowanie złożonych scenariuszy z prostych testów-kroków oraz
możliwość wykorzystania asercji w~aplikacji poza testami.
Możliwe jest także wykorzystanie Ash do testowania mobilnych wersji stron internetowych. 

\section{Wprowadzenie}
\subsection{Hybrydowe aplikacje mobilne}
Mówiąc o natywnej aplikacji mobilnej mamy na myśli apliakcję tworzoną z myślą o konkretnej platformie (Android, iOS, itp.) przy użyciu narzuconych przez twórcę platformy narzędzi (Java, Objective-C, itd.). Aplikacje mobile web są to z kolei aplikacje webowe zoptymalizowane po kątem urządzeń natywnych.
Adobe PhoneGap oraz jej odpowiednik o otwartym źródle Apache Cordova, to
dwie popularne biblioteki pozwalajace na tworzenie hybrydowych aplikacji
mobilnych, tj. aplikacji które łączą w sobie zalety aplikacji natywnych oraz aplikacji
typu mobile web. Zasada działania tego typu aplikacji jest w założeniu prosta i
polega na wykorzystaniu komponentów, które dalej nazywać będziemy WebView.
Kontrolki te są dostępne na każdej nowoczesnej platoformie i pozwalają nam na
wyświetalnie stron internetowych z wnętrza natywnych aplikacji mobilnych. 
W momencie startu aplikacja hybrydowa tworzy WebView oraz ładuje do niego zasoby z określonego adresu (lokalnego lub zdalnego), które są wyświetlane w WebView. Najczęściej tym zasobem jest aplikacja stworzona przy użyciu technologii webowych.     

% tabelki w LaTeX-u też da się robić, chociaż najczytelniej to może nie wygląda
% więcej na temt tabelek można poczytac np. na http://en.wikibooks.org/wiki/LaTeX/Tables
\begin{center}
    \begin{tabular}{ |  p{2cm} | p{4cm} | p{4cm} | p{4cm} |}
    \hline
   			& Aplikacje natywne        & Aplikacje hybrydowe 	& Aplikacje mobile web 			\\ \hline
    Narzędzia	& Zależne od platformy	&   Narzędzia webowe	& 	Narzędzia webowe			\\ \hline
    Instalacja	&        Wymagana               &	 Wymagana         	& 		Nie wymagana		\\ \hline
    Wydajność	& 	Bardzo dobra		&Ograniczona przez przeglądarkę&Ograniczona przez przeglądarkę\\ \hline
    Możliwość monetyzacji
    			&  Pobranie, reklamy, subskrybcje  & Pobranie, reklamy, subskrybcje & Reklamy, subskrybcje \\ \hline
    Dostęp do urządzenia
    			& 	Tak, pełen  	         &Najważniejsze funkcjonalności& Ograniczony        \\ \hline
    \end{tabular}
\end{center}

Takie podejście ma wiele zalet. Dzięki temu, że do tworzenia aplikacji hybrydowych wykorzystywane są technologie
znane z zastosowań intenetowych, koszt tworzenia aplikacji i czas dostarczenia
gotowego rozwiązania na rynek są znacznie zredukowane. Aplikacje tego typu są też
z założenia wieloplatformowe. Używając PhoneGap z tego samego kodu źródłowego
możemy stworzyć aplikacje na platformę Android, iOS, Blackberry, WebOS,
Windows Phone, Symbiana i Bada. Dodatkowo popularność narzędzi internetowych
ułatwia znalezienie właściwych programistów.

\subsection{Wady podejścia hybrydowego}
Największymi wadami podejścia hybrydowego są słabsza wydajność, błędy
pojawiające się tylko na określonych urządzeniach oraz różnorodne oczekiwania
użytkowników różnych platform. Użytkownik instalując aplikację hybrydową w taki sam sposób co 
natywną spodziewa się, że będzie ona działać niczym natywna. Tymczasem
dodatkowy narzut hybrydy, często w połączeniu z niechlujnie przygotowaną i nie
zoptymalizowaną aplikacją, prowadzi do mniej responsywnego interfejsu
użytkownika i w konskewencji do frustracji użytkownika. Częstym problemem przy
tworzeniu aplikacji mobilnych są błędy pojawiające się tylko na określonych
urządzeniach. Przy podejściu hybrydowym problem jest o tyle bardziej widoczny, że
najczęściej tworzymy rozwiązanie które ma być przenośne nie tylko pomiędzy urządzeniami w ramach jednej platformy, ale także między platformami. Co z oczywistych względów zwiększa przestrzeń urządzeń, które trzeba uzwględnić. Trzecim problemem hybryd jest kwestia
tworzenia interfejsu użytkownika. Dostawcy systemów operacyjnych dla urządzeń
mobilnych publikują zalecenia, co do tego jak powinien wyglądać interfejs aplikacji
działającej pod danym systemem. Zalecenia te są specyficzne dla platformy i często
wzajemnie się wykluczają. Przygotowanie jednej szaty graficznej i jednego interfejsu
może zostać źle odebrane przez użytkowników spodziewających się wyglądu
dostosowanego do platformy. Niestety stworzenie kilku wersji interfejsu jest dużo
bardziej pracochłonne i skomplikowane, niweczy także podstawową zaletę apliakcji
hybrydowych – przenośność. 
Są to poważne niedostatki, ale braki te można zniwelować poprzez skuteczne narzędzia.

\subsection{Propozycje rozwiązania problemów}
Ash ma za zadanie pomóc rozwiązać problemy zwiąne z wydajnością aplikacji oraz z
błędami pojawiającymi się tylko na określonych konfiguracjach sprzętowych. Aby
zapewnić wysoką wydajność działania aplikacji konieczny jest rygor w tworzeniu
wydajnego oprogramowania oraz możliwość testowania jej w sytuacjach które mogę
uwydatnić problemy z wydajnością. Dzięki funkcyjnemu podejściu do testowania
oprogramowania Ash pozwala na zbieranie informacji o responsywności interfejsu
oraz realistycznie symulować scenariusze dużego obciążenia. Programista
korzystający z Ash ma możliwość zdefiniowania w scenariuszach maksymalnego
czasu przebeigu testu, jeśli test nie zakończy się w założonym czasie test nie
powiedzie się. Niska responsywność interfejsu traktowana jest na równi z błędami
logiki czy prezentacji. Wymusza to na twórcy zadbanie o szybkość reakcji interfejsu.
Ash oferuje także możliwość chwilowego wyłączenia lub opóźnienia dostępu do
sieci. Sytuacje w których dostępność do internetu jest ograniczona są dość częste,
jednak niewiele aplikacji jest pisana biorąc to pod uwagę, jeszcze mniej jest
testowana pod tym kątem. Bardzo często problemy z dotępem objawiają się kiepską
wydajnością interfejsu lub długimi przestojami na ekranach ładowania. Ash pozwala
twórcom świadomie zmierzyć się z tym problemem. Ash wyposażony jest w
mechanizmy pozwalajace na łatwe uruchomienie na aplikacji na wielu urządzeniach
naraz, fizycznych jak i wirtualnych, także w zdalnych lokalizacjach. Sprawia to, że
użytkownicy są mają możliwość masowego uruchamiania testów na wszystkich
dostępnych im urządzeniach oraz są bardziej skłonni skorzystać z testów w czaseie
swojej pracy. Możliwość zdalnego uruchomienia testów daje możliwość stworzenia
rozproszonej bazy urządzeń, co pozwoli na testowanie także nietypowych
konfiguracji.

\subsection{Dlaczego Apache Cordova}
Jako bazę do implementacji wybrałem Apache Cordova. Apache Cordova jest to
wersja PhoneGap, którą korporacja Adobe (właściel praw do PhoneGap) udostępniła
fundacji Apache. Od tego momentu ten wariant technologii udostępniany jest
zasadzie otwartego źródła. Pomimo tego oba projekty są niemal identyczne, a jedyna
różnica faktyczna różnica między nimi jest natury prawnej. Głównym powodem
wyboru tej technologii jest spory udziała w rynku hybrydowych aplikacji mobilnych,
dynamizm rozwoju oraz liczna społeczność.

\subsection{Propozycje rozwiązania problemów}
Ash ma za zadanie pomóc rozwiązać problemy zwiąne z wydajnością aplikacji oraz z błędami pojawiającymi się tylko na określonych konfiguracjach sprzętowych. Aby zapewnić wysoką  wydajność działania aplikacji konieczny jest rygor w tworzeniu wydajnego oprogramowania oraz możliwość testowania jej w sytuacjach które mogę uwydatnić problemy z wydajnością. Dzięki funkcyjnemu podejściu do testowania oprogramowania Ash pozwala na zbieranie informacji o responsywności interfejsu oraz realistycznie symulować scenariusze dużego obciążenia. Programista korzystający z Ash ma możliwość zdefiniowania w scenariuszach maksymalnego czasu przebeigu testu, jeśli test nie zakończy się w założonym czasie test nie powiedzie się. Niska responsywność interfejsu traktowana jest na równi z błędami logiki czy prezentacji. Wymusza to na twórcy zadbanie o szybkość reakcji interfejsu. Ash oferuje także możliwość chwilowego wyłączenia lub opóźnienia dostępu do sieci. Sytuacje w których dostępność do internetu jest ograniczona są dość częste, jednak niewiele aplikacji jest pisana biorąc to pod uwagę, jeszcze mniej jest testowana pod tym kątem. Bardzo często problemy z dotępem objawiają się kiepską wydajnością interfejsu lub długimi przestojami na ekranach ładowania. Ash pozwala twórcom świadomie zmierzyć się z tym problemem. Ash wyposażony jest w mechanizmy pozwalajace na łatwe uruchomienie na aplikacji na wielu urządzeniach naraz, fizycznych jak i wirtualnych, także w zdalnych lokalizacjach. Sprawia to, że użytkownicy są mają możliwość masowego uruchamiania testów na wszystkich dostępnych im urządzeniach oraz są bardziej skłonni skorzystać z testów w czaseie swojej pracy. Możliwość zdalnego uruchomienia testów daje możliwość stworzenia rozproszonej bazy urządzeń, co pozwoli na testowanie także nietypowych konfiguracji. 

\subsection{Propozycje rozwiązania problemów}
Jako bazę do implementacji wybrałem Apache Cordova. Apache Cordova jest to wersja PhoneGap, którą korporacja Adobe (właściel praw do PhoneGap) udostępniła fundacji Apache. Od tego momentu ten wariant technologii udostępniany jest zasadzie otwartego źródła. Pomimo tego oba projekty są niemal identyczne, a jedyna różnica faktyczna różnica między nimi jest natury prawnej. Głównym powodem wyboru tej technologii jest spory udziała w rynku hybrydowych aplikacji mobilnych, dynamizm rozwoju oraz liczna społeczność.

\section{Sposób użycia}
\subsection{Hello World}

\subsection{Getting Started}

\subsection{Filozofia}

\section{Przegląd funkcjonalności}
\subsection{Zmiana położenia ekranu}

\subsubsection{Opis}
Pozwala ustawić ekran w określonym położeniu - landscape lub portrait.

\subsubsection{Sposób użycia}

\subsection{Dostęp do sieci}

\subsubsection{Opis}
Daje możliwość manipulowania dostępem do sieci - opóźniać reakcję, zatrzymywać oraz przywracać połączenie.

\subsubsection{Sposób użycia}

\subsection{Dostęp do systemu plików}

\subsubsection{Opis}
Daje dostęp do systemu plików - pozwala na symulowanie dostępu do lokalnych plików oraz ich uploadu.

\subsubsection{Sposób użycia}

\subsection{Symulacja ruchu}

\subsubsection{Opis}
Ash dostarcza API umożliwiające symulowanie położenia lub ruchu między dwoma punktami.

\subsubsection{Sposób użycia}

\subsection{Obsługa przycisku powrotu}

\subsubsection{Opis}
Ta funkcjonalność symuluje naciśnięcie przycisku powrotu na telefonie.

\subsubsection{Sposób użycia}

\section{Zastosowania}

\subsection{Problemy związane z obecnymi rozwiązaniami}

\subsection{Ash jako rozwiązanie}

\subsection{RWE - real world example}

\section{Architektura}

\subsection{Spojrzenie z lotu ptaka}

\subsection{Promise}

\subsection{Organizacja testów. Run oraz Play}

\subsection{Native bridge. Integracja z Cordova}

\section{Implementacja}

\subsection{Zmiana położenia ekranu}

\subsection{Dostęp do sieci}

\subsection{Dostęp do systemu plików}

\subsection{Symulacja ruchu}

\subsection{Obsługa przycisku powrotu}

\section{Narzędzia oraz środowisko programisty}

\subsection{Rozpraszanie wykonania testów}

\subsection{Tworzenie laboratorium urządzeń}

\section{Dalsze kierunki rozwoju}

\end{document}