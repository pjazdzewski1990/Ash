\documentclass[a4paper]{article}
\usepackage[pdftex]{graphicx}
% oprócz formatu papieru zdefiniowanego na początku (A4) możemy
% jeszcze określić wielkość „obszaru zadrukowanego”:
\usepackage[body={17cm,24cm}]{geometry}
%%%%%%%%%%%%%%%%%%%%%%%%%%%%%%%%%%%%%%%%%%%%%%%%%%%%%%%%%%%%
% aby ustalić, że plik źródłowy używa kodowania UTF-8:
%-----------------------------------------------------------
\usepackage[utf8]{inputenc}
% jeśli jedynie chcemy wykorzystywać polskie (lub inne) 
% znaki diakrytyczne wystarczy użyć pakietu "fontenc"
% z parametrem "OT4":
%\usepackage[OT4]{fontenc}
% jeśli piszemy po polsku, należy zamiast powyższego użyć:
\usepackage{polski}
%%%%%%%%%%%%%%%%%%%%%%%%%%%%%%%%%%%%%%%%%%%%%%%%%%%%%%%%%%%%
\begin{document}
\section{Streszczenie}
% Poniższy akapit zawiera przykłady użycia „niełamliwej spacji” w celu uniknięcia
% „wiszących liter” na końcach wiersza – zjawisko uważne za błąd w polskiej typografii.
Poniższa praca zawiera opis biblioteki „Ash” służącej do funkcjonalnego testowania
hybrydowych aplikacji mobilnych stworzonych przy użyciu Adobe PhoneGap lub
Apache Cordova. Ash pozwala na testowanie zachowania się aplikacji w~różnorodnych
realistycznych przypadkach, takich jak poruszanie się użytkownika,
obrót ekranu i~inne. Dzięki wykorzystaniu hybrydowego charakteru aplikacji
możliwa jest emulacja zachowania, co wpływa na większy realizm testów. Innymi
zaletami bliblioteki są elastyczna struktura testów, która ułatwia utrzymywanie
testów oraz budowanie złożonych scenariuszy z prostych testów-kroków oraz
możliwość wykorzystania asercji w~aplikacji poza testami.

\section{Wprowadzenie}
\subsection{Hybrydowe aplikacje mobilne}
Adobe PhoneGap oraz jej odpowiednik o otwartym źródle Apache Cordova, to
dwie popularne biblioteki pozwalajace na tworzenie hybrydowych aplikacji
mobilnych tj. aplikacji które łączą w sobie zalety aplikacji natywnych oraz aplikacji
typu mobile web. Zasada działania tego typu aplikacji jest w założeniu prosta i
polega na wykorzystaniu komponentów, które dalej nazywać będziemy WebView.
Kontrolki te są dostępne na każdej nowoczesnej platoformie i pozwalają nam na
wyświetalnie stron internetowych z wnętrza natywnych aplikacji mobilnych.

% tabelki w LaTeX-u też da się robić, chociaż najczytelniej to może nie wygląda
% więcej na temt tabelek można poczytac np. na http://en.wikibooks.org/wiki/LaTeX/Tables
\begin{center}
    \begin{tabular}{ | l | l | l | p{5cm} |}
    \hline
   				& Aplikacje natywne & Aplikacje hybrydowe 	& Aplikacje mobile web 	\\ \hline
    Narzędzia	& 					&						& 						\\ \hline
    Instalacja	& 					&						& 						\\ \hline
    Wydajność	& 					&						& 						\\ \hline
    Możliwość monetyzacji
    			& 					&						& 						\\ \hline
    Dostęp do urządzenia
    			& 					&						& 						\\ \hline
    \end{tabular}
\end{center}

Dzięki temu, że do tworzenia aplikacji hybrydowych wykorzystywane są technologie
znane z zastosowań intenetowych, koszt tworzenia aplikacji i czas dostarczenia
gotowego rozwiązania na rynek są znacznie zredukowane. Aplikacje tego typu są też
z założenia wieloplatformowe. Używając PhoneGap z tego samego kodu źródłowego
możemy stworzyć aplikacje na platformę Android, iOS, Blackberry, WebOS,
Windows Phone, Symbiana i Bada. Dodatkowo popularność narzędzi internetowych
ułatwia znalezienie właściwych programistów.

\subsection{Wady podejścia hybrydowego}
Największymi wadami podejścia hybrydowego są słabsza wydajność, błędy
pojawiające się tylko na określonych urządzeniach oraz różnorodne oczekiwania
użytkowników różnych platform. Użytkownik instalując aplikację hybrydową jak
natywną spodziewa się, że będzie ona działać niczym natywna. Tymczasem
dodatkowy narzut hybrydy, często w połączeniu z niechlujnie przygotowaną i nie
zoptymalizowaną aplikacją, prowadzi do mniej responsywnego interfejsu
użytkownika i w konskewencji do frustracji użytkownika. Częstym problemem przy
tworzeniu aplikacji mobilnych są błędy pojawiające się tylko na określonych
urządzeniach. Przy podejściu hybrydowym problem jest o tyle bardziej widoczny, że
najczęściej tworzymy kod nie tylko pod jedną platformę, tak jak ma to miejsce przy
podejściu natywnym, ale pod wiele. Trzecim problemem hybryd jest kwestia
tworzenia interfejsu użytkownika. Dostawcy systemów operacyjnych dla urządzeń
mobilnych publikują zalecenia, co do tego jak powinien wyglądać interfejs aplikacji
działającej pod danym systemem. Zalecenia te są specyficzne dla platformy i często
wzajemnie się wykluczają. Przygotowanie jednej szaty graficznej i jednego interfejsu
może zostać źle odebrane przez użytkowników spodziewających się wyglądu
dostosowanego do platformy. Niestety stworzenie kilku wersji interfejsu jest dużo
bardziej pracochłonne i skomplikowane, niweczy także podstawową zaletę apliakcji
hybrydowych – przenośność.

\subsection{Propozycje rozwiązania problemów}
Ash ma za zadanie pomóc rozwiązać problemy zwiąne z wydajnością aplikacji oraz z
błędami pojawiającymi się tylko na określonych konfiguracjach sprzętowych. Aby
zapewnić wysoką wydajność działania aplikacji konieczny jest rygor w tworzeniu
wydajnego oprogramowania oraz możliwość testowania jej w sytuacjach które mogę
uwydatnić problemy z wydajnością. Dzięki funkcyjnemu podejściu do testowania
oprogramowania Ash pozwala na zbieranie informacji o responsywności interfejsu
oraz realistycznie symulować scenariusze dużego obciążenia. Programista
korzystający z Ash ma możliwość zdefiniowania w scenariuszach maksymalnego
czasu przebeigu testu, jeśli test nie zakończy się w założonym czasie test nie
powiedzie się. Niska responsywność interfejsu traktowana jest na równi z błędami
logiki czy prezentacji. Wymusza to na twórcy zadbanie o szybkość reakcji interfejsu.
Ash oferuje także możliwość chwilowego wyłączenia lub opóźnienia dostępu do
sieci. Sytuacje w których dostępność do internetu jest ograniczona są dość częste,
jednak niewiele aplikacji jest pisana biorąc to pod uwagę, jeszcze mniej jest
testowana pod tym kątem. Bardzo często problemy z dotępem objawiają się kiepską
wydajnością interfejsu lub długimi przestojami na ekranach ładowania. Ash pozwala
twórcom świadomie zmierzyć się z tym problemem. Ash wyposażony jest w
mechanizmy pozwalajace na łatwe uruchomienie na aplikacji na wielu urządzeniach
naraz, fizycznych jak i wirtualnych, także w zdalnych lokalizacjach. Sprawia to, że
użytkownicy są mają możliwość masowego uruchamiania testów na wszystkich
dostępnych im urządzeniach oraz są bardziej skłonni skorzystać z testów w czaseie
swojej pracy. Możliwość zdalnego uruchomienia testów daje możliwość stworzenia
rozproszonej bazy urządzeń, co pozwoli na testowanie także nietypowych
konfiguracji.

\subsection{dlaczego phonegap, phonegap vs cordova}
Jako bazę do implementacji wybrałem Apache Cordova. Apache Cordova jest to
wersja PhoneGap, którą korporacja Adobe (właściel praw do PhoneGap) udostępniła
fundacji Apache. Od tego momentu ten wariant technologii udostępniany jest
zasadzie otwartego źródła. Pomimo tego oba projekty są niemal identyczne, a jedyna
różnica faktyczna różnica między nimi jest natury prawnej. Głównym powodem
wyboru tej technologii jest spory udziała w rynku hybrydowych aplikacji mobilnych,
dynamizm rozwoju oraz liczna społeczność.
\end{document}